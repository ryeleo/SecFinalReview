\documentclass{report}
\usepackage[margin=0.25in]{geometry}


\begin{document}
% Chapter 4: review questions 4.1-4.7, problems 4.3, 4.4, 4.5, 4.8
{\bf R.4.1} MAC provides a static, provably safe access control system. DAC
trusts the admission and rejection of access rights to user's as opposed to
administrators and thus it is not provably secure.

{\bf R.4.2} RBAC relates to MAC in that users are assigned a role by the
administrator and have access to only the system objects that people with that
role have access rights to. RBAC varies from both DAC and MAC in that the access
rights of a role are strictly applied to groups of people who are given that
role, while DAC and MAC traditionally are determined on a per-user/person basis.

{\bf R.4.3} 
Process: A process running with given permissions
User: A user
Group: A set of users

{\bf R.4.4} A subject is the actor, it acts on objects. The object is the entity
of interest, it is acted upon.

{\bf R.4.5} An access right is an indicator that a particular subject can act on
a particular object.

{\bf R.4.6}
ACL: An object and the rights that each subject has to act on it.
Ticket: A subject and the rights it has to act on each object.

{\bf R.4.7}
A protection domain is a grouping of access rights that a particular set of
subjects have. One simple example of this is Kernel access rights as opposed to
User access rights; these could be considered to be split into different
protection domains.

%{\bf 4.3}
%{\bf 4.4}
%{\bf 4.5}
%{\bf 4.8}

% Chapter 6: review questions 6.1, 6.7, 6.9, problems 6.7, 6.10, 6.12
{\bf R.6.1} Malware can propagate through 1. social engineering attacks, 2.
Vulnerability Exploits, 3. Infected Content.

{\bf R.6.7} A drive-by-download is a web browser exploit (or potentially other
internet application I think) that is triggered upon visiting a certain
web-site. This differs from a worm in that the drive-by-download does not
actively propagate itself once it has infected a host, it continues to sit
dormant on the web server waiting for other unsuspecting users to visit the
page.

{\bf R.6.9} 
Backdoor: A mechanism that allows bypassing of a security check.

Bot: A machine whose computational/network resources have been remotely taken
over by a hacker.

Keylogger: Software that captures the key-strokes of a user. 

Spyware: Software that collects information about users using the infected
machine and sends the information off to the hackers server. \texttt{This
does/can include a keylogger}

Rootkit: A set of tools which are installed with root access that will provide a
hacker a returning backdoor, as well as other tools the hacker might need when
accessing the infected machine. This is done with root access, so at the kernel
level, and cannot be detected by the operating system because it is a part of
the operating system essentially. \texttt{This does/can include all of the above}



% Chapter 8: review questions 8.1-8.9, 8.13, problems 8.6, 8.8
{\bf R.8.1}
{\bf R.8.2}
{\bf R.8.3}
{\bf R.8.4}
{\bf R.8.5}
{\bf R.8.6}
{\bf R.8.7}
{\bf R.8.8}
{\bf R.8.9}
{\bf R.8.13}



% Chapter 9: review questions 9.1-9.5, 9.11, 9.13, problems 9.4, 9.5
{\bf R.9.1}
{\bf R.9.2}
{\bf R.9.3}
{\bf R.9.4}
{\bf R.9.5}
{\bf R.9.13}
{\bf R.9.11}



% Chapter 10: review questions 10.1-10.9, problem 10.10
{\bf R.10.1}
{\bf R.10.2}
{\bf R.10.3}
{\bf R.10.4}
{\bf R.10.5}
{\bf R.10.6}
{\bf R.10.7}
{\bf R.10.8}
{\bf R.10.9}



% Chapter 11: review questions 11.4-11.7, 11.11, 11.13, problem 11.4
{\bf R.11.4}
{\bf R.11.5}
{\bf R.11.6}
{\bf R.11.7}
{\bf R.11.11}
{\bf R.11.13}



% Chapter 12: review questions 12.3-12.7, 12.17, problems 12.1, 12.4, 12.5
{\bf R.12.3}
{\bf R.12.4}
{\bf R.12.5}
{\bf R.12.6}
{\bf R.12.7}
{\bf R.12.17}


% Chapter 13: review questions 13.2, 13.4, 3.8, 13.9, 13.13, problems 13.1, 13.2, 13.4, 13.9
{\bf R.13.2}
{\bf R.13.4}
{\bf R.13.8}
{\bf R.13.9}
{\bf R.13.13}

\end{document}
