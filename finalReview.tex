\documentclass{report}
\usepackage[margin=0.25in]{geometry}


\begin{document}
% Chapter 4: review questions 4.1-4.7, problems 4.3, 4.4, 4.5, 4.8
{\bf R.4.1} MAC provides a static, provably safe access control system. DAC
trusts the admission and rejection of access rights to user's as opposed to
administrators and thus it is not provably secure.

{\bf R.4.2} RBAC relates to MAC in that users are assigned a role by the
administrator and have access to only the system objects that people with that
role have access rights to. RBAC varies from both DAC and MAC in that the access
rights of a role are strictly applied to groups of people who are given that
role, while DAC and MAC traditionally are determined on a per-user/person basis.

{\bf R.4.3} 
Process: A process running with given permissions
User: A user
Group: A set of users

{\bf R.4.4} A subject is the actor, it acts on objects. The object is the entity
of interest, it is acted upon.

{\bf R.4.5} An access right is an indicator that a particular subject can act on
a particular object.

{\bf R.4.6}
ACL: An object and the rights that each subject has to act on it.
Ticket: A subject and the rights it has to act on each object.

{\bf R.4.7}
A protection domain is a grouping of objects to a single set of access rights.
That is, rather than give every object in the group an individual set of access
rights, we see that they are all identical and just assign them to the same
group.

{\bf 4.3}
{\bf 4.4}
{\bf 4.5}
{\bf 4.8}

% Chapter 6: review questions 6.1, 6.7, 6.9, problems 6.7, 6.10, 6.12
% Chapter 8: review questions 8.1-8.9, 8.13, problems 8.6, 8.8
% Chapter 9: review questions 9.1-9.5, 9.11, 9.13, problems 9.4, 9.5
% Chapter 10: review questions 10.1-10.9, problem 10.10
% Chapter 11: review questions 11.4-11.7, 11.11, 11.13, problem 11.4
% Chapter 12: review questions 12.3-12.7, 12.17, problems 12.1, 12.4, 12.5
% Chapter 13: review questions 13.2, 13.4, 3.8, 13.9, 13.13, problems 13.1, 13.2, 13.4, 13.9

\end{document}
